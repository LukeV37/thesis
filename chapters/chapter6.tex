\begingroup
\clearpage% Manually insert \clearpage
\let\clearpage\relax% Remove \clearpage functionality
\vspace*{-16pt}% Insert needed vertical retraction
\chapter[STANDARD MODEL HADRONIC HIGGS ANALYSIS]{STANDARD MODEL HADRONIC HIGGS ANALYSIS}
\endgroup

\section{Key Contributions}

\begin{itemize}
  \item Contributed to two ATLAS Higgs analysis in the Higgs decaying to B/C Quarks.
  \item For the Higgs production mode with associated vector boson, I ran studies in the CC, single lepton channel to include a new low $p_T$ region $75 GeV < p_T < 150 GeV$ which ultimately increased the containing power on the background and for the first time allowed for $5\sigma$ for the $Z\rightarrow cc$ process and decreased the limited for $H\rightarrow cc$ by about 5\%.
  \item For the Higgs production mode of Vector Boson Fusion, I assisted at nearly every part of the analysis, from sample production, flavor tagging optimization, adversarial neural network classifier, region definitions, fit studies, and fit combinations. For the first time, this analysis measured $3\sigma$ in the VBF $H\rightarrow bb$ channel and reported a limit on the VBF $H\rightarrow cc$ process. 
\end{itemize}

\section{Higgs Production at the LHC}

A landmark achievement in experimental particle physics was obtained at the LHC in 2012 with the discovery of the Higgs Boson. This particle is a quantization of the Higgs field which is responsible for mass generation in the standard model, which up until then was a crucial missing piece. The Higgs field has a wine bottle shape that leads to spontaneous symmetry breaking which is a crucial part of the Higgs mechanism giving rise to the mass of vector bosons. Since its discovery, many of the Higgs properties have been studied including its mass, spin, coupling to gauge bosons, and coupling to third generation fermions. As of date, all experimental observations are in line with the Standard Model predictions.

\begin{figure}[h]
	\centering
	\includegraphics[width=\linewidth]{figures/chapter6/Higgs_Production.png}
	\caption{}
	\label{fig:HiggsProduction}
\end{figure}

The Higgs boson has four unique production modes, shown in Figure \ref{fig:HiggsProduction}. In current LHC conditions with a center of mass energy at 13 TeV, the Hanbook of LHC Higgs cross section https://arxiv.org/abs/1610.07922 shows that $\sigma_{ggF} \approx 50 pb$, $\sigma_{VBF} \approx 4pb$, $\sigma_{VH} \approx 2.5 pb$, and $\sigma_{ttH} \approx 0.5 pb$. Each of these channels has their own unique backgrounds. The branching ratio of the Higgs boson is shown in Figure \ref{fig:HiggsBranchingRatios}. Notice the largest branching ratio is $H \rightarrow bb$ at around 58\%. Unfortunately, to take advantage of this channel an analysis will need to suppress the dominant hadronic multi-jet background at the LHC. Currently, ATLAS has focused large efforts on the VH channel, as it allows leptonic channels originating from the associated vector boson.

\begin{figure}[h]
	\centering
	\includegraphics[width=0.4\linewidth]{figures/chapter6/Higgs_Branching_Ratios.png}
	\caption{}
	\label{fig:HiggsBranchingRatios}
\end{figure}

\section{Associated Higgs Production with Vector Boson Decaying to B/C Quarks}

Despite the smaller cross section, the VH production channel has yielded the best measurements to date within the ATLAS experiment for $H \rightarrow bb$ measurement and $H \rightarrow cc$ limits. This is primarily due to the use of lepton and MET triggers that instantly suppress the large QCD multi-jet background. The leading backgrounds for this channel are $t\bar{t}$ and V+jets production. However, Boosted Decision Trees, BDTs, are used to learn the difference between signal and background. Regions are defined by channel (0,1,2 leptons), pT slice, and BDT score. I joined this analysis in quite a mature stage, so I helped out with fit studies. Under the guidance of Yanhui Ma, a postdoc in our group, I studied the impact of including a new low $p_T$ region $75 GeV < p_T < 150 GeV$ which had a vast amount of statistics. Due to the high statistics, this region caused significant pulls in the fit model, so we ran studies to decorrelated various nuissance parameters to add degrees of freedom into the fit model. This regions helped constrain the background modeling and overall improve the sensitivity to signal. In addition to measuring the VH channel, the VZ channel was used as a cross check analysis to validate the VH channel. 

\subsection{Statistical Methods using Likelihood Fits}

In order to measure signal strength, $\mu$, we perform a statistical procedure to match the theoretical modeling, MC, to real world data. A likelihood function $\mathcal{L}(\mu,\theta)$ is constructed as the product of Poisson probability terms over the bins of the input distributions where $x_i$ is data counted in bin i and $s_i$ and $b_i$ are signal and background predictions, respectively. The parameter of interest (POI), or signal strength $\mu$, is extracted by maximising the likelihood function $\mathcal{L}$. The effects of systematic uncertainties are quantified as $\theta$, also called Nuissance Parameters NP, and are often assumed to be constrained by gaussian distributions. $\alpha_j$ is known as the global measurement data. Usually assumed mean=0 and var=1. The likelihood fit function is shown below.

\begin{equation}
	\mathcal{L}(\mu,\theta|x)=\prod_i Pois(x_i|\mu \cdot s_i(\theta)+b_i(\theta)) \prod_j C(\alpha_j|\theta_j)
\end{equation}

%\begin{figure}
%	\centering
%	\includegraphics[width=0.7\linewidth]{figures/chapter6/Likelihood_Fit_Function.png}
%	\caption{}
%	\label{fig:VBF_Diagrams}
%\end{figure}

In the VH analysis, the global likelihood fit contained 59 signal regions with 27 for $H \rightarrow bb$ and 32 for $H \rightarrow cc$ and 97 control regions. Signal regions are plotted against the BDT score to maximally separate the signal from the background. Control regions are used to constrain background normalization factors, particularly the floating normalization used to constrain $t\bar{t}$ and V+jets. By the time I joined this analysis, many of these regions were already finalized. However, Yanhui Ma tasked me with including a new signal region into the fit model. 

\begin{figure}[h]
	\centering
	\begin{subfigure}{.49\textwidth}
		\centering
		\includegraphics[width=0.8\linewidth]{figures/chapter6/New_Region_Hcc.png}
		\caption{}
		\label{fig:sub1}
	\end{subfigure}%
	\begin{subfigure}{.49\textwidth}
		\centering
		\includegraphics[width=0.8\linewidth]{figures/chapter6/New_Region_Zcc.png}
		\caption{}
		\label{fig:sub2}
	\end{subfigure}
	\caption{}
	\label{fig:ANN_Scores}
\end{figure}

\subsection{Results}

After including these regions into the fit model, I ran likelihood fits using the WSMaker framework to determine the impact of the new regions. In the table, we can see that the new region decreased the limit of the $H \rightarrow cc$ channel by 4\% expected and 8\% observed. However, there was a large improvement in the $Z \rightarrow cc$ channel where we found gains of 15\% expected and nearly 20\% observed. Note, that these fits were performed on blinded data, so data with a BDT score greater than 0.2 was excluded from the fit. The purpose of blinding data it common practice to make sure that the analysers are not making biased decisions in the fit model that would artificially inflate the signal measurement. Therefore decisions regarding the fit model should be motivated by unblinded data, and the final unblinded result is revealed a the very end. After unblinding, the VZ channel yield signal strength of:
\begin{align}
	\mu^{bb}_{VZ}&=0.93^{+0.13}_{-0.11}=0.92 \pm 0.05 (Stat)^{+0.12}_{-0.10}(Syst) \\
	\mu^{cc}_{VZ}&=0.98^{+0.25}_{-0.22}=0.98 \pm 0.13 (Stat)^{+0.22}_{-0.18}(Syst)
\end{align}

These measurements result in an observed significance in the $VZ,Z\rightarrow bb$ process of greater than 10 standard deviations. For the $VZ,Z\rightarrow cc$ process, the observed (expected) significance over the null hypothesis is 5.2 (5.3) standard deviations. The low pT region in the single lepton channel was a crucial region for $5\sigma$ discovery of $VZ,Z\rightarrow cc$ process. 

The final unblinded result in the VH channel yield signal strength of:
\begin{align}
	\mu^{bb}_{VH}&=0.92^{+0.16}_{-0.15}=0.92 \pm 0.10 (Stat)^{+0.13}_{-0.11}(Syst) \\
	\mu^{cc}_{VH}&=1.0^{+5.4}_{-5.2}=1.0^{+4.0}_{-3.9} (Stat)^{+3.7}_{-3.5}(Syst)
\end{align}
For the $VH,H\rightarrow bb$ process, the observed (expected) significance over the null hypothesis is 7.4 (8.0) standard deviations. For the $VH,H\rightarrow cc$ process, the observed (expected) upper limit at 95\% confidence level is 11.5 (10.6) times the standard model predictions.

\begin{figure}
	\centering
	\includegraphics[width=0.6\linewidth]{figures/chapter6/New_Region_Results.png}
	\caption{}
	\label{fig:VBF_Diagrams}
\end{figure}

\section{Vector Boson Fusion Higgs Production Decaying to B/C Quarks}

Recently, both ATLAS and CMS have reported evidence for the coupling of the Higgs boson to muons, providing the first direct probe of Higgs boson interactions with second-generation fermions. However, in the quark sector, one of the key properties that remains to be explored is the coupling of the Higgs boson to charm quarks.

The search for Higgs boson decay into a charm quark-antiquark pair (H $\rightarrow$ cc) remains challenging because of the Higgs boson's smaller Yukawa coupling to charm quarks than to bottom quarks, the overwhelming quantum chromodynamics (QCD) multijet backgrounds, and the challenges in experimentally distinguishing charm jets from light flavor or bottom jets. The most stringent observed (expected) limits are 11.5 (10.6) times the SM prediction, from ATLAS in the VH production mode.

This analysis presents a search for H $\rightarrow$ cc by exploiting the second-largest Higgs boson production mode, vector-boson fusion (VBF). In parallel, the analysis also targets H $\rightarrow$ bb decays in VBF production. The bottom-quark Yukawa coupling plays a central role in the Higgs sector, and while H $\rightarrow$ bb has been measured in associated-production modes (VH and ttH), it has not yet been observed in the VBF production mode.

\begin{figure}[h]
	\centering
	\includegraphics[width=0.7\linewidth]{figures/chapter6/VBF_Signal_and_Background.png}
	\caption{}
	\label{fig:VBF_Diagrams}
\end{figure}

The VBF production mode is characterized by the presence of two forward jets with large invariant mass and rapidity separation, produced via the exchange of a virtual vector boson between the two initial-state quarks. A new inclusive trigger exploiting this topology was deployed in ATLAS partway through 2018, during Run 2 of the LHC. It achieves an efficiency for VBF H $\rightarrow$ bb events comparable to that of the b-tagging-based trigger used in the Run-2 VBF H $\rightarrow$ bb analysis, while also enabling the first search for the VBF H $\rightarrow$ cc process. The H $\rightarrow$ cc search utilizes proton-proton collision data collected by the ATLAS detector at two center-of-mass energies: 37.5 fb$^{-1}$ at $\sqrt{s} = 13$ TeV from part of the 2018 Run-2 period and 51.5 fb$^{-1}$ at $\sqrt{s} = 13.6$ TeV during 2022-2023 in Run 3, yielding a total integrated luminosity of 89.0 fb$^{-1}$. The H $\rightarrow$ bb process is measured only with the Run-3 data, allowing direct combination with the analysis performed on full Run-2 dataset. The analysis strategy follows previous work, with improvements in multivariate techniques, the adoption of the inclusive VBF trigger, and the use of a novel transformer-based flavor-tagging algorithm for jet flavor identification.

\subsection{Event Selection}

Simulated VBF signal events were generated using Monte Carlo methods. The simulations include both vector-boson fusion and gluon-gluon fusion processes. Background events are estimated using a data-driven method that accounts for the overwhelming QCD multijet production. The hadronic resonant background from vector boson plus jets is modeled using simulation techniques, with contributions from both QCD and electroweak processes. Other Higgs boson production modes are not considered due to their negligible contributions.

Jet reconstruction uses the particleflow algorithm, which combines calorimeter energy deposits and inner-detector tracks into particleflow objects that are clustered into jets using the anti-k$_t$ algorithm with a radius parameter R = 0.4. To mitigate the impact of pileup, dedicated jet-vertex-tagging (JVT) requirements are applied. Jets with pseudorapidity $|\eta| < 2.5$ are required to have $p_T > 20$ GeV, whereas jets with $|\eta| > 2.5$ must satisfy $p_T > 30$ GeV.

Selected events must pass the inclusive VBF trigger, which requires at least two jets: one with $p_T > 70$ GeV and $|\eta| < 3.2$, and another with $p_T > 50$ GeV and $|\eta| < 4.9$. Additional dijet requirements are imposed, namely $|\Delta\eta_{12}| > 4$, $|\Delta\phi_{12}| < 2$, and $M_{12} > 1000$ (1100) GeV for Run 3 (Run 2). Jets not identified as VBF candidate jets are considered for the selection of Higgs candidate jets (denoted by J$_{h1}$ and J$_{h2}$), originating from the H $\rightarrow$ cc or H $\rightarrow$ bb decay. Both J$_{h1}$ and J$_{h2}$ are required to have $p_T > 20$ GeV and $|\eta| < 2.5$. To suppress shape distortions in the background invariant-mass distributions, the Higgs candidate jets must also satisfy $p_{T,h1h2} > 150$ GeV and $|\Delta\phi_{h1h2}| < 2$. If multiple jets satisfy these criteria, the pair maximizing $p_{T,h1h2}$ is selected.
\begin{figure}
	\centering
	\includegraphics[width=0.5\linewidth]{figures/chapter6/tabaux_01.png}
	\caption{}
	\label{fig:ANN_Scores}
\end{figure}

The two Higgs candidate jets must be either b-tagged or c-tagged. Flavor identification is performed using the transformer-based GN2 algorithm, which exploits charged-track information and achieves better separation of b-jets, c-jets, and light flavor jets than the previous DL1r tagger. The calibrated tagger output defines working points corresponding to various b- and c-jet efficiency targets, using established calibration methods consistent with those described in previous works. To identify b-jets, this analysis uses the 77\% b-tagging-efficiency working point, which has a c-jet (light flavor jet) efficiency of 6\% (0.24\%). Jets failing this requirement are classified as c-tagged if they satisfy the 50\% c-tagging-efficiency working point, which has a b-jet (light flavor jet) efficiency of 13\% (5.3\%). Two tighter working points, corresponding to 30\% and 10\% c-jet efficiencies, provide additional inputs to the neural network.
\begin{figure}
	\centering
	\includegraphics[width=0.4\linewidth]{figures/chapter6/figaux_15.png}
	\caption{}
	\label{fig:ANN_Scores}
\end{figure}

Events are categorized into three channels: the bb Run-3 channel, where both Higgs candidate jets are b-tagged, and the cc Run-2 and Run-3 channels, where both are c-tagged.

\subsection{Adversarial Neural Network Training}

The overall analysis strategy is similar to previous work. A neural-network classifier is used to separate events into different analysis regions, and the signal yield is extracted from a simultaneous fit to the Higgs candidate mass (M$_{h1h2}$) distribution in all regions. The dominant Non-res. bkgd is estimated from data; its shape is determined from the lowest-score region, assuming it is largely identical across all regions. To realize this strategy, an adversarial neural network (ANN), similar to that in previous work, is trained separately for each analysis channel in order to decorrelate the neural-network classifier output from M$_{h1h2}$. The networks are implemented in PyTorch and optimized using the Adam optimizer. Inputs comprise event and jet kinematics and jet flavor-tagging information. For training, signal events are taken from VBF MC samples, while background events are drawn from data sidebands on either side of the Higgs boson mass window ($100 < M_{h1h2} < 140 GeV$) but within the range $50 < M_{h1h2} < 200 GeV$. The number of regions and their boundaries are optimized for maximum sensitivity, resulting in six signal regions per channel (SR1-SR6), ordered by decreasing signal purity, which ranges from 2\% to 0.02\%. The least pure region, SR6, is used as a control region.

\begin{figure}
	\centering
	\includegraphics[width=0.35\linewidth]{figures/chapter6/ANN.png}
	\caption{}
	\label{fig:ANN}
\end{figure}

The H $\rightarrow$ cc contamination in the H $\rightarrow$ bb channel is negligible, whereas the expected H $\rightarrow$ bb yield in the H $\rightarrow$ cc channel increases from about 40\% to 100\% of the H $\rightarrow$ cc signal across SR1-SR5 and is approximately 1.4 times larger in SR6. The expected ggF-to-VBF yield ratio increases from about 1\% in SR1 to 8\% in SR5, and reaches about 45\% in SR6 in all channels; the ggF ANN distribution closely resembles that of the backgrounds, explaining the weak sensitivity of the measurement to ggF.

\begin{figure}
	\centering
	\includegraphics[width=0.5\linewidth]{figures/chapter6/tab_02.png}
	\caption{}
	\label{fig:ANN_Features}
\end{figure}

\begin{figure}
	\centering
	\begin{subfigure}{.32\textwidth}
		\centering
		\includegraphics[width=1\textwidth]{figures/chapter6/fig_02a.png}
		\caption{}
		\label{fig:sub1}
	\end{subfigure}%
	\begin{subfigure}{.32\textwidth}
		\centering
		\includegraphics[width=1\textwidth]{figures/chapter6/fig_02b.png}
		\caption{}
		\label{fig:sub2}
	\end{subfigure}
	\begin{subfigure}{.32\textwidth}
		\centering
		\includegraphics[width=1\textwidth]{figures/chapter6/fig_02c.png}
		\caption{}
		\label{fig:njets}
	\end{subfigure}
	\caption{}
	\label{fig:ANN_Scores}
\end{figure}

\subsection{Statistical Fit Model}

Signal production rates are extracted from a binned maximum-likelihood fit to the M$_{h1h2}$ distributions in all regions and channels, using 5 GeV bins over the range 50 < M$_{h1h2}$ < 200 GeV, resulting in a total of 30 bins per region. The expected H $\rightarrow$ cc and H $\rightarrow$ bb yields are scaled by independent signal-strength modifiers, $\mu_{cc}$ and $\mu_{bb}$, defined as the ratios of the observed to the SM-expected signal yields. These modifiers are treated as parameters of interest (POIs) and are extracted by maximizing the likelihood.

The analysis employs six signal regions per channel (SR1-SR6), ordered by decreasing signal purity, which ranges from 2\% to 0.02\%. The least pure region, SR6, serves as a control region. The dominant background, non-resonant QCD multijet production, is estimated from data using a data-driven method. The shape of the background in high-score regions (SR1-SR5) is determined from the signal-depleted low-score region (SR6). This approach is motivated by the adversarial neural network design, which aims to decorrelate the reconstructed mass and classifier score.

The background shape in each signal region is modeled using a data-driven approach. For each channel, the background shape in the high-score regions is taken from the signal-depleted low-score region (SR6). This strategy effectively constrains the multijet background shape in the high-score regions since SR6 contains a much larger number of events and is signal-depleted relative to SR1-SR5. The shape is determined by the classifier score distribution in the sidebands of SR6, which is then applied to all high-score regions.

Systematic uncertainties are incorporated through nuisance parameters (NPs). The dominant systematic uncertainties include those from the non-resonant background bias uncertainty and the effects of limited Monte Carlo sample sizes, particularly those from the QCD V+jets samples. Detector-related uncertainties, primarily those associated with jets, are also included. The bias and residual uncertainties are applied to account for potential distortions in the signal extraction due to the neural network training on sideband data.

The residual uncertainties capture potential correlations: for each signal region, an uncertainty is derived from M$_{h1h2}$ sidebands and applied across the full mass range. A fit to the ratio of the M$_{h1h2}$ distributions in the sidebands of SR6 to those in each of SR1-SR5 is performed, where a linear function adequately describes the observed differences; higher-order polynomials were tested but did not provide any significant improvement. The fitted slopes are implemented as unconstrained shape variations in the likelihood, while the offsets are absorbed by the normalization factors.

The bias uncertainties quantify possible distortions of the M$_{h1h2}$ spectrum within the Higgs mass window ($100 < M_{h1h2} < 140 GeV$) that may arise because the ANN is trained only on sideband events and may bias the signal extraction. They are estimated by performing signal-plus-background fits on multijet MC samples, which are treated as data after being reweighted using sideband data information to improve the modelling. The resulting bias in the fitted signal strength, computed as the quadrature sum of the extracted signal yield and its statistical uncertainty, is taken as a systematic uncertainty. The bias amounts to about 27\%-51\% of the expected statistical uncertainty in the H $\rightarrow$ bb signal and 26\%-56\% of that in the H $\rightarrow$ cc signal. Both the bias and residual uncertainties are applied in SR1-SR5, treated as uncorrelated across regions, leading to 15 nuisance parameters for each set.

\begin{figure}
	\centering
	\begin{subfigure}{.32\textwidth}
		\centering
		\includegraphics[width=1\textwidth]{figures/chapter6/figaux_12a.png}
		\caption{}
		\label{fig:sub1}
	\end{subfigure}%
	\begin{subfigure}{.32\textwidth}
		\centering
		\includegraphics[width=1\textwidth]{figures/chapter6/figaux_12b.png}
		\caption{}
		\label{fig:sub2}
	\end{subfigure}
	\begin{subfigure}{.32\textwidth}
		\centering
		\includegraphics[width=1\textwidth]{figures/chapter6/figaux_12c.png}
		\caption{}
		\label{fig:njets}
	\end{subfigure}
	\begin{subfigure}{.32\textwidth}
		\centering
		\includegraphics[width=1\textwidth]{figures/chapter6/figaux_12d.png}
		\caption{}
		\label{fig:sub1}
	\end{subfigure}%
	\begin{subfigure}{.32\textwidth}
		\centering
		\includegraphics[width=1\textwidth]{figures/chapter6/figaux_12e.png}
		\caption{}
		\label{fig:sub2}
	\end{subfigure}
	\begin{subfigure}{.32\textwidth}
		\centering
		\includegraphics[width=1\textwidth]{figures/chapter6/figaux_12f.png}
		\caption{}
		\label{fig:njets}
	\end{subfigure}
	\caption{}
	\label{fig:ANN_Scores}
\end{figure}

\begin{figure}
	\centering
	\begin{subfigure}{.32\textwidth}
		\centering
		\includegraphics[width=1\textwidth]{figures/chapter6/figaux_13a.png}
		\caption{}
		\label{fig:sub1}
	\end{subfigure}%
	\begin{subfigure}{.32\textwidth}
		\centering
		\includegraphics[width=1\textwidth]{figures/chapter6/figaux_13b.png}
		\caption{}
		\label{fig:sub2}
	\end{subfigure}
	\begin{subfigure}{.32\textwidth}
		\centering
		\includegraphics[width=1\textwidth]{figures/chapter6/figaux_13c.png}
		\caption{}
		\label{fig:njets}
	\end{subfigure}
	\begin{subfigure}{.32\textwidth}
		\centering
		\includegraphics[width=1\textwidth]{figures/chapter6/figaux_13d.png}
		\caption{}
		\label{fig:sub1}
	\end{subfigure}%
	\begin{subfigure}{.32\textwidth}
		\centering
		\includegraphics[width=1\textwidth]{figures/chapter6/figaux_13e.png}
		\caption{}
		\label{fig:sub2}
	\end{subfigure}
	\begin{subfigure}{.32\textwidth}
		\centering
		\includegraphics[width=1\textwidth]{figures/chapter6/figaux_13f.png}
		\caption{}
		\label{fig:njets}
	\end{subfigure}
	\caption{}
	\label{fig:ANN_Scores}
\end{figure}

\begin{figure}
	\centering
	\includegraphics[width=0.5\linewidth]{figures/chapter6/tabaux_02.png}
	\caption{}
	\label{fig:Yields}
\end{figure}

\subsection{Analysis Results}

Figure 1 shows the observed data together with the signal and background estimates in each bin of the Higgs candidate mass discriminant. All SRs are weighted by ln(1 + S/B) and summed; here S and B denote the total background and signal yields, respectively. The distributions are shown separately for the H $\rightarrow$ bb and H $\rightarrow$ cc channels, with the signal definition restricted to one Higgs boson decay mode at a time.

\begin{figure}[h]
	\centering
	\begin{subfigure}{.49\textwidth}
		\centering
		\includegraphics[width=0.8\textwidth]{figures/chapter6/fig_01a.png}
		\caption{}
		\label{fig:sub1}
	\end{subfigure}%
	\begin{subfigure}{.49\textwidth}
		\centering
		\includegraphics[width=0.8\textwidth]{figures/chapter6/fig_01b.png}
		\caption{}
		\label{fig:sub2}
	\end{subfigure}
	\caption{}
	\label{fig:ANN_Scores}
\end{figure}

The measured signal strengths for the H $\rightarrow$ bb and H $\rightarrow$ cc signals are:

\begin{align}
	\mu_{bb} &= 0.97^{+0.57}_{-0.50} = 0.97 \pm 0.45 \, (\text{stat.}) ^{+0.35}_{-0.22} (\text{syst.}) \\
	\mu_{cc} &= 18 \pm 13 = 18 \pm 12 \, (\text{stat.}) \pm 7 \, (\text{syst.})
\end{align}

with a correlation of -4.5\% between the two measurements. The measurement of $\mu_{bb}$ yields an observed (expected) significance of 2.0 (2.1) standard deviations over the background-only prediction. No significant excess over the background-only hypothesis is observed in the search for H $\rightarrow$ cc. An upper limit on $\mu_{cc}$ is calculated using the CLs method, with the profile-likelihood ratio as the test statistic. The observed 95\% confidence level (CL) upper limit on $\mu_{cc}$ is 41, while a limit of 28 is expected in the case of no H $\rightarrow$ cc process. The result is also interpreted in the 'kappa framework' by reparameterizing $\mu_{cc}$ in terms of the charm Yukawa coupling modifier $\kappa_c$, defined as $\kappa_c^2 = \Gamma_c/\Gamma_{SMc}$, where $\Gamma_c$ is the partial decay width into a charm-quark pair. Assuming that all other couplings follow their SM predictions and only SM decays are considered, the parameterization is:
\begin{equation}
	\kappa_c = \frac{\mu_{cc}}{\sqrt{BR_{SM}(H \rightarrow cc)}}
\end{equation}

where $BR_{SM}(H \rightarrow cc)$ is the SM H $\rightarrow$ cc branching fraction. The expected 95\% CL constraint on the modifier is $|\kappa_c| < 11.6$. No observed constraint can be set at 95\% CL due to the large observed value of $\mu_{cc}$. Instead, a 68\% CL interval of $2.3 < |\kappa_c| < 18.4$ is obtained. An additional fit, performed by considering only the VBF process as signal, while fixing the ggF contribution to its SM prediction within uncertainties, yields a nearly identical result for both the H $\rightarrow$ bb and H $\rightarrow$ cc measurements.

The contributions of each uncertainty source to the total uncertainty in the fitted $\mu_{cc}$ and $\mu_{bb}$ values are summarized in Table 1. Both measurements are statistically limited. In both cases, the dominant systematic uncertainties arise from the Non-res. bkgd bias uncertainty and the effects of limited MC sample size (particularly those from the QCD V+jets samples), followed by the detector-related uncertainties, primarily those associated with jets.

\begin{figure}
	\centering
	\includegraphics[width=0.4\linewidth]{figures/chapter6/tab_01.png}
	\caption{}
	\label{fig:Yields}
\end{figure}

\begin{figure}
	\centering
	\includegraphics[width=0.65\linewidth]{figures/chapter6/tabaux_03.png}
	\caption{}
	\label{fig:Yields}
\end{figure}

\subsection{Combinations}

Two combined analyses are performed: one combining the H $\rightarrow$ cc result with the Run-2 VH-channel results, and another combining the Run-3 H $\rightarrow$ bb result with the Run-2 result in the VBF channel and a recently reported Run-2 photon-associated VBF (VBF+$\gamma$) result. For the H $\rightarrow$ cc combination, $\mu_{bb}$ is also allowed to float. For the H $\rightarrow$ bb combination, the H $\rightarrow$ cc channels from this analysis are excluded and $\mu_{cc}$ is fixed to 1, with the ggF contributions fixed to their SM predictions within uncertainties. For both combinations, jet-related experimental uncertainties are correlated, except for those related to jet flavor tagging, where different algorithms and calibration procedures are used; background modeling uncertainties are treated as uncorrelated, while theory uncertainties in signal production cross-sections and branching ratios are correlated.

From the first combination, the observed $\mu_{cc,Comb}$ signal strength relative to the SM prediction is 3.3$^{+5.2}_{-5.0}$, assuming SM VBF, ggF, and VH production rates. The observed (expected) 95\% CL upper limit on $\mu_{cc,Comb}$ is 13 (10), corresponding to an observed (expected) constraint of $|\kappa_c| < 4.7 (3.9)$. This represents an 8\% improvement in the expected limit and a 5\% improvement in the expected $|\kappa_c|$ constraint relative to the VH-only result. The observed result is slightly weaker than that from the VH channel, owing to an upward fluctuation in $\mu_{cc,Comb}$ driven by the VBF contribution. From the second combination, the observed $\mu_{bb,VBF}$ signal strength relative to the SM prediction is 0.87 $\pm$ 0.28, corresponding to an observed (expected) significance of 3.2 (3.6) standard deviations, thus providing the first evidence for the VBF H $\rightarrow$ bb process. The p-value for compatibility of the signal-strength parameters among the three analyses is 65\%, and 95\% between the Run-2 and Run-3 VBF H $\rightarrow$ bb analyses.

\begin{figure}
	\centering
	\begin{subfigure}{.49\textwidth}
		\centering
		\includegraphics[width=\linewidth]{figures/chapter6/figaux_03.png}
		\caption{}
		\label{fig:sub1}
	\end{subfigure}%
	\begin{subfigure}{.49\textwidth}
		\centering
		\includegraphics[width=\linewidth]{figures/chapter6/figaux_04.png}
		\caption{}
		\label{fig:sub2}
	\end{subfigure}
	\caption{}
	\label{fig:ANN_Scores}
\end{figure}

\begin{figure}
	\centering
	\begin{subfigure}{.32\textwidth}
		\centering
		\includegraphics[width=\linewidth]{figures/chapter6/figaux_05.png}
		\caption{}
		\label{fig:sub1}
	\end{subfigure}%
	\begin{subfigure}{.32\textwidth}
		\centering
		\includegraphics[width=\linewidth]{figures/chapter6/figaux_06.png}
		\caption{}
		\label{fig:sub2}
	\end{subfigure}
	\begin{subfigure}{.32\textwidth}
		\centering
		\includegraphics[width=\linewidth]{figures/chapter6/figaux_07.png}
		\caption{}
		\label{fig:sub2}
	\end{subfigure}
	\caption{}
	\label{fig:ANN_Scores}
\end{figure}

\section{Conclusion}

In summary, this thesis presents the first search for SM Higgs boson decay into a pair of charm quarks in the VBF production mode in the ATLAS experiment at the LHC, together with a simultaneous measurement of Higgs boson decay into a bottom quark-antiquark pair. The VBF H $\rightarrow$ cc search is enabled by a new trigger designed to exploit the VBF topology. The observed signal strengths relative to the SM predictions are $\mu_{bb} = 0.97^{+0.57}_{-0.50}$ and $\mu_{cc} = 18 \pm 13$. For the H $\rightarrow$ cc process, this corresponds to an observed (expected) 95\% CL upper limit of 41 (28) on $\mu_{cc}$. When combined with previous Run-2 results, these measurements improve the expected constraint on the Higgs-charm coupling and both the expected and observed constraints on the Higgs-bottom coupling.



