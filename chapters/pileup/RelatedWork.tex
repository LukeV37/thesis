\section{Related Work}\hfill

Machine learning approaches for High Energy Physics problems are gaining traction from multiple perspectives with advancements in neural networks~\cite{he2023high,Barman_2024,Larkoski_2020}. Although there are several sub-problems in High Energy Physics, such as jet flavor tagging~\cite{qu2022particle`}, top tagging, generative event modeling~\cite{kansal2021particle}, unfolding, anomaly detection, and calibration that have been explored with machine learning, methods to analyze \emph{PileUp} have been mostly overlooked in the existing work. One of such interesting problems is mitigating pile-up particles and correcting jet mass and energy for events that occur at the Large-Hadron Collider~\cite{komiske2017pileup}. \\ 

Existing pileup mitigation techniques for the ATLAS and CMS experiments have focused on solving the binary classification problem for either jets or tracks. ATLAS uses algorithms to classify jets using classifiers such as kNN algorithms~\cite{ATLAS-CONF-2014-018} which rely on constructing high level variables on a per jet basis. CMS uses an algorithm to classify tracks using a statistical $\chi^2$ approach through the PUPPI~\cite{Bertolini_2014} algorithm classifies pileup at the particle level. However, these methods fail to incorporate correlations between jets at the Event level for Event-Wide context. Correlations exists between jet originating from Hard Scatter processes.  \\

Graph Neural Networks have also been studied for for pileup mitigation, however, it is non-trivial how to form a graph in the context of an event. Its unfeasible to connect all particles in an event due to computation limitations, and connecting all tracks within a jet can confuse the model unless edge weights are assigned properly for HS and PU particles. Dynamic edge convolutions have been studied, but this can lead to long training times due to calculating kNN in latent space. Attention, on the other hand, gives a highly parallelizable algorithm to automatically determine weights between objects and update node representations accordingly.