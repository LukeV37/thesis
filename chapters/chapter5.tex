\begingroup
\clearpage% Manually insert \clearpage
\let\clearpage\relax% Remove \clearpage functionality
\vspace*{-16pt}% Insert needed vertical retraction
\chapter[CALIBRATION OF GN2Xv01 BOOSTED JET TAGGER]{CALIBRATION OF GN2Xv01 BOOSTED JET TAGGER}
\endgroup

\section{Key Contributions}

\begin{itemize}
  \item Derived ttbar enriched region using ATLAS AnalysisTop framework in ATHENA r24.
  \item Validated GN2X discriminant in ttbar enriched region and quantified ttbar mistag rate.
  \item Derived scale factors to account for the differences between MC and data.

\end{itemize}

\section{Boosted Jet Tagging with Graph and Attention Neural Networks}

\lipsum[1]

\begin{align}
	D^{GN2X}_{Hbb} &= ln \left( \frac{p_{Hbb}}{f_{Hcc} \cdot p_{Hcc} + f_{top} \cdot p_{top} + (1-f_{Hcc}-f_{top})\cdot p_{QCD}} \right) \\
	D^{GN2X}_{Hcc} &= ln \left( \frac{p_{Hcc}}{f_{Hbb} \cdot p_{Hbb} + f_{top} \cdot p_{top} + (1-f_{Hbb}-f_{top})\cdot p_{QCD}} \right)
\end{align}

\lipsum[1]

\begin{figure}
	\centering
	\includegraphics[width=0.5\linewidth]{figures/chapter5/DXbb.png}
	\caption{}
	\label{fig:Yields}
\end{figure}

\section{ttbar Mistag Calibration}

\lipsum[1]

\begin{equation}
	SF=\frac{\epsilon^{data}}{\epsilon^{MC}}
\end{equation}

\lipsum[1]

\begin{align}
	N^{data}_{tag} &= \mu \cdot \frac{\epsilon^{data}}{\epsilon^{MC}} \cdot N^{ttbar}_{tag} + N^{other}_{tag} \\
	N^{data}_{untag} &= \mu \cdot \frac{1-\epsilon^{data}}{1-\epsilon^{MC}} \cdot N^{ttbar}_{untag} + N^{other}_{untag}
\end{align}

Or equivalently the formula can be written in terms of scale factors:

\begin{align}
	N^{data}_{tag} &= \mu \cdot SF \cdot N^{ttbar}_{tag} + N^{other}_{tag} \\
	N^{data}_{untag} &= \mu \cdot \frac{1-\epsilon^{MC} \cdot SF}{1-\epsilon^{MC}} \cdot N^{ttbar}_{untag} + N^{other}_{untag}
\end{align}

\lipsum[1]

\subsection{Event Selection}

\lipsum[1]

\begin{figure}
	\centering
	\includegraphics[width=0.7\linewidth]{figures/chapter5/Reco_Event.png}
	\caption{}
	\label{fig:Yields}
\end{figure}

\lipsum[1]

\begin{table}[h]
	\centering
	\resizebox{0.7\textwidth}{!}{%
		\begin{tabular}{ccc}
			\hline
			\textbf{Cut Applied} & \textbf{Electron Channel} & \textbf{Muon Channel} \\
			\hline
			Initial Events                   & 8,911,000 & 8,911,000 \\
			Primary Vertex                   & 8,910,952 & 8,910,952 \\
			Reconstruction                   & 1,220,162 & 1,220,162 \\
			Lepton $p_T > 25$ GeV            & 374,572   & 400,935   \\
			Tight lepton $> 70$ GeV          & 96,701    & 79,744    \\
			Pflow jet $p_T > 25$ GeV         & 94,826    & 77,999    \\
			UFO large-R jet $p_T > 200$ GeV  & 53,462    & 41,511    \\
			MET                              & 44,089    & 33,959    \\
			\hline
		\end{tabular}%
	}
	\caption{Cutflow table for electron and muon selections.}
	\label{tab:cutflow}
\end{table}

%\begin{figure}
%	\centering
%	\includegraphics[width=0.7\linewidth]{figures/chapter5/Cutflow.png}
%	\caption{}
%	\label{fig:Yields}
%\end{figure}

\lipsum[1-3]

\begin{figure}
	\centering
	\begin{subfigure}{.24\textwidth}
		\centering
		\includegraphics[width=\linewidth]{figures/chapter5/probe_jet_pt.png}
		\caption{}
		\label{fig:sub1}
	\end{subfigure}%
	\begin{subfigure}{.24\textwidth}
		\centering
		\includegraphics[width=\linewidth]{figures/chapter5/probe_jet_eta.png}
		\caption{}
		\label{fig:sub2}
	\end{subfigure}
	\begin{subfigure}{.24\textwidth}
		\centering
		\includegraphics[width=\linewidth]{figures/chapter5/probe_jet_phi.png}
		\caption{}
		\label{fig:sub2}
	\end{subfigure}
	\begin{subfigure}{.24\textwidth}
	\centering
	\includegraphics[width=\linewidth]{figures/chapter5/probe_jet_mass.png}
	\caption{}
	\label{fig:sub2}
	\end{subfigure}
	\caption{}
	\label{fig:ANN_Scores}
	\begin{subfigure}{.24\textwidth}
		\centering
		\includegraphics[width=\linewidth]{figures/chapter5/probe_jet_mass_300_400.png}
		\caption{}
		\label{fig:sub1}
	\end{subfigure}%
	\begin{subfigure}{.24\textwidth}
		\centering
		\includegraphics[width=\linewidth]{figures/chapter5/probe_jet_mass_400_500.png}
		\caption{}
		\label{fig:sub2}
	\end{subfigure}
	\begin{subfigure}{.24\textwidth}
		\centering
		\includegraphics[width=\linewidth]{figures/chapter5/probe_jet_mass_500_600.png}
		\caption{}
		\label{fig:sub2}
	\end{subfigure}
	\begin{subfigure}{.24\textwidth}
		\centering
		\includegraphics[width=\linewidth]{figures/chapter5/probe_jet_mass_600_1000.png}
		\caption{}
		\label{fig:sub2}
	\end{subfigure}
	\caption{}
	\label{fig:ANN_Scores}
\end{figure}

\subsection{Constructing Fit Regions}

\lipsum[1]

\begin{figure}
	\centering
	\begin{subfigure}{.32\textwidth}
		\centering
		\includegraphics[width=\linewidth]{figures/chapter5/CR_pt_200_postfit.png}
		\caption{}
		\label{fig:sub1}
	\end{subfigure}%
	\begin{subfigure}{.32\textwidth}
		\centering
		\includegraphics[width=\linewidth]{figures/chapter5/CR_pt_250_postfit.png}
		\caption{}
		\label{fig:sub2}
	\end{subfigure}
	\begin{subfigure}{.32\textwidth}
		\centering
		\includegraphics[width=\linewidth]{figures/chapter5/CR_pt_300_postfit.png}
		\caption{}
		\label{fig:sub2}
	\end{subfigure}
	\begin{subfigure}{.32\textwidth}
		\centering
		\includegraphics[width=\linewidth]{figures/chapter5/CR_pt_400_postfit.png}
		\caption{}
		\label{fig:sub2}
	\end{subfigure}
	\begin{subfigure}{.32\textwidth}
	\centering
	\includegraphics[width=\linewidth]{figures/chapter5/CR_pt_500_postfit.png}
	\caption{}
	\label{fig:sub2}
	\end{subfigure}
	\begin{subfigure}{.32\textwidth}
	\centering
	\includegraphics[width=\linewidth]{figures/chapter5/CR_pt_600_postfit.png}
	\caption{}
	\label{fig:sub2}
	\end{subfigure}
	\caption{}
	\label{fig:ANN_Scores}
\end{figure}

\lipsum[1]

\begin{figure}
	\centering
	\begin{subfigure}{.32\textwidth}
		\centering
		\includegraphics[width=\linewidth]{figures/chapter5/SR_pt_200_postfit.png}
		\caption{}
		\label{fig:sub1}
	\end{subfigure}%
	\begin{subfigure}{.32\textwidth}
		\centering
		\includegraphics[width=\linewidth]{figures/chapter5/SR_pt_250_postfit.png}
		\caption{}
		\label{fig:sub2}
	\end{subfigure}
	\begin{subfigure}{.32\textwidth}
		\centering
		\includegraphics[width=\linewidth]{figures/chapter5/SR_pt_300_postfit.png}
		\caption{}
		\label{fig:sub2}
	\end{subfigure}
	\begin{subfigure}{.32\textwidth}
		\centering
		\includegraphics[width=\linewidth]{figures/chapter5/SR_pt_400_postfit.png}
		\caption{}
		\label{fig:sub2}
	\end{subfigure}
	\begin{subfigure}{.32\textwidth}
		\centering
		\includegraphics[width=\linewidth]{figures/chapter5/SR_pt_500_postfit.png}
		\caption{}
		\label{fig:sub2}
	\end{subfigure}
	\begin{subfigure}{.32\textwidth}
		\centering
		\includegraphics[width=\linewidth]{figures/chapter5/SR_pt_600_postfit.png}
		\caption{}
		\label{fig:sub2}
	\end{subfigure}
	\caption{}
	\label{fig:ANN_Scores}
\end{figure}

\lipsum[1]

\subsection{Results}

\lipsum[1]

\begin{figure}
	\centering
	\begin{subfigure}{.49\textwidth}
		\centering
		\includegraphics[width=\linewidth]{figures/chapter5/scale_factors_mc16.png}
		\caption{}
		\label{fig:sub1}
	\end{subfigure}%
	\begin{subfigure}{.49\textwidth}
		\centering
		\includegraphics[width=\linewidth]{figures/chapter5/scale_factors_mc20.png}
		\caption{}
		\label{fig:sub2}
	\end{subfigure}
	\caption{}
	\label{fig:ANN_Scores}
\end{figure}

\lipsum[1]

\section{conclusion}

\lipsum[1]


