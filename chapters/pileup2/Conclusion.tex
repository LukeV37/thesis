\section{Conclusion}\hfill

We suggest a novel model, \myname{}, to address the effect of the pileup interactions on physics studies at the current and future LHC conditions. The model makes use of a set of transformers with self- and cross-attention with the input of track and jet parameters in the event. The proposed architecture allows the model to learn kinematic correlations arising from physics processes, which helps to recover the hard scatter process properties. As the output, \myname{} predicts the fractions of jet energy and jet mass due to pileup. The model was trained and tested using simulated di-Higgs datasets. It was shown that the model is capable of recovering the value and resolution of the Higgs boson mass in high-pileup conditions, where initially the Higgs boson mass peak is not possible to observe. \myname{} also scales well with the expected increase of the pileup level at the LHC and HL-LHC.

%The LHC prepares to upgrade to the HL-LHC phase which brings significantly more pileup contamination to hard scatter jets. To mitigate the effects of high pileup conditions, we introduce $E_{frac}$ and $M_{frac}$ to perform direct identification, energy, and mass correction of hard scatter jets. We train the proposed method on a regression task to directly predict $E_{frac}$ and $M_{frac}$ through a diHiggs simulated dataset. The model uses transformer encoders using self and cross attention to enrich representations of jet features using all possible correlations between jet and tracks within the context of an event. We show that this model scales well in high pileup conditions, and through extensive analysis we show that the expected Higgs mass resonance can be restored and the effects of pileup can be mitigated. Lastly, $E_{frac}$ and $M_{frac}$ can be used as physically significant features and increase performance and a direct binary classification task using an attention architecture. 

%In conclusion, we presented the following contributions.
%\begin{enumerate}
%    \item We proposed a first-of-its-kind pileup prediction modeled as a regression problem.
%    \item We proposed a cross-attention based neural network architecture that utilizes jets and tracks information for pileup fraction detection.
%    \item We showed with extensive analysis that the proposed method outperforms the baseline approaches.
%    \item We also showed that the predictions from the proposed approach also assist with physics processes.
%\end{enumerate}