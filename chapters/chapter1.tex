
\begingroup
\clearpage% Manually insert \clearpage
\let\clearpage\relax% Remove \clearpage functionality
\vspace*{-16pt}% Insert needed vertical retraction
\chapter[INTRODUCTION]{INTRODUCTION}\label{chap:intro}
\endgroup

\section{Outline}

\begin{itemize}
  \item general introduction, breif history of QM and particle physics
  \item Explain HEP research focus during my time as PhD such higgs theory and analysis.
  \item How does this research fit into global picture.
  \item Modern statistical analysis methods in HEP (ML: Adversarial, Graph, Attention).
  \item HL-LHC mention/motivation
\end{itemize}

\begin{figure}
  \includegraphics[width=\linewidth]{figures/chapter1/LHC-and-HL-LHC-Schedule.png}
  \caption{The upgrade shcedule for HL-LHC operations. \cite{Zerlauth:2024PD}}
\end{figure}

\begin{figure}[t]
    \centering
    \begin{subfigure}{.49\textwidth}
      \centering
      \includegraphics[width=\linewidth]{figures/chapter1/Event_30_mu60.png}
      \caption{}
      \label{fig:Efrac1d_mu60}
    \end{subfigure}\hfill
    \begin{subfigure}{.49\textwidth}
      \centering
      \includegraphics[width=\linewidth]{figures/chapter1/Event_30_mu200.png}
      \caption{}
      \label{fig:Efrac2d_mu60}
    \end{subfigure}\hfill
    \caption{A Phoenix Event Display\cite{phoenix} depicting an event in the ATLAS inner tracking system in the HL-LHC conditions at $\langle\mu\rangle=60$ (left) and $\langle\mu\rangle=200$ (right) for signal (blue) and background (red) particles with $p_T>1.0$ GeV.}
    \label{fig:PhoenixDisplay}
\end{figure}

