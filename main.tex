\documentclass[12pt]{osuthesis}
\usepackage[letterpaper, margin=1in]{geometry}

%%%%%%%%%%%%%%%%%%%%%%%%%%%%%%%%%%% NOTICE %%%%%%%%%%%%%%%%%%%%%%%%%%%%%%%%%%%
% BASIC INSTRUCTIONS:
% Find the sections marked with a NOTICE of this form and fill in the relevant
% information. You will then put the substance of your thesis in the separate
% chapter TeX files. Remember, you will always compile THIS file.

% Warnings on using this file:
% - osuthesis.cls is not currently compatible with the hyperref package.
% - Do not include package amsthm, \qed and such commands are already
%   defined in osuthesis

%%%%%%%%%%%%%%%%%%%%%%%%%%%%%%%%%%%%%%%%%%%%%%%%%%%%%%%%%%%%%%%%%%%%%%%%%%%%%%


\usepackage{amsmath,amssymb}
\usepackage{graphicx}
  \graphicspath{ {figures/} }
\usepackage{comment}
\usepackage{enumerate}
\usepackage{mathrsfs}
\usepackage{fancyhdr}
\usepackage{changepage}

\usepackage{afterpage}
\usepackage[titles]{tocloft}

\usepackage{chngcntr}

%%% User Defined Packages %%%
\usepackage[hidelinks]{hyperref}
\usepackage{subcaption}
\usepackage{multirow}
\usepackage{booktabs}
\usepackage{bm}
\usepackage[normalem]{ulem}
\usepackage{lipsum}


%%%%%%%%%%%%%%%%%%%%%%%%%%%%%%%%%%% NOTICE %%%%%%%%%%%%%%%%%%%%%%%%%%%%%%%%%%%
% If you want your tables and/or figures to be numbered within chapter and section,
% change the following two commands to \counterwithin instead of \counterwithout:
\counterwithout{figure}{chapter}
\counterwithout{table}{chapter}

\renewcommand{\cftchapaftersnum}{.}

%%%%%%%%%%%%%%%%%%%%%%%%%%%%%%%%%%% NOTICE %%%%%%%%%%%%%%%%%%%%%%%%%%%%%%%%%%%
% Make any of these widths wider if needed
\renewcommand*\cftchapnumwidth{3.5em}
\renewcommand*\cftsecnumwidth{3.4em}
\renewcommand*\cftsubsecnumwidth{4em}
% The second argument (1.5em) is the space before the
% I.1 entry for the figure/table
% The third argument (4em) is the space to allow for the I.1
% style entry.
%%%%%%%%%%%%%%%%%%%%%%%%%%%%%%%%%%%%%%%%%%%%%%%%%%%%%%%%%%%%%%%%%%%%%%%%%%%%%%


\makeatletter
%\renewcommand*\l@figure{\@dottedtocline{1}{1.5em}{4em}}
%\renewcommand*\l@table{\@dottedtocline{1}{1.5em}{4em}}
\makeatother


\numberwithin{equation}{section}

%%%%%%%%%%%%%%%%%%%%%%%%%%%%%%%%%%% NOTICE %%%%%%%%%%%%%%%%%%%%%%%%%%%%%%%%%%%
% Put your shortcuts for math symbols etc. that you want to use here in this
% section. They will be available in every chapter file afterwards.
% Some example shortcuts are included below.
\def\A{{\mathbb A}}
\def\C{{\mathbb C}}
\def\D{{\mathbb D}}
\def\K{{\mathbb K}}
\def\L{{\mathbb L}}
\def\N{{\mathbb N}}
\def\Q{{\mathbb Q}}
\def\R{{\mathbb R}}
\def\Z{{\mathbb Z}}
\def\E{{\mathbb E}}
\def\T{{\mathbb T}}
\def\P{{\mathbb P}}
\def\S{{\mathbb S}}
\def\nn{\mathbf{n}}
\newcommand{\cE}{\mathcal{E}}
\newcommand{\cF}{\mathcal{F}}
\def\eps{\varepsilon}
\def\vp{\varphi}
\newcommand{\ds}{\displaystyle{}}
\newcommand{\abs}[1]{\left\lvert#1\right\rvert}
\newcommand{\BigO}[1]{\ensuremath{\operatorname{O}\bigl(#1\bigr)}}
\newcommand{\defeq}{\mathrel{\mathop:}=}
\newcommand{\Var}{\mathrm{Var}}
\newcommand{\Cov}{\mathrm{Cov}}
\DeclareMathOperator{\csch}{csch}
\newcommand{\overbar}[1]{\mkern 1.5mu\overline{\mkern-1.5mu#1\mkern-1.5mu}\mkern 1.5mu}
\newcommand{\Sp}{\mathbb S}
\newcommand{\cp}{\operatorname{cap}}
\newcommand{\dist}{\operatorname{dist}}
\newcommand{\interi}{\operatorname{int}}
\newcommand{\norm}[1]{\left\lVert#1\right\rVert} % norm: double vertical bars
\newcommand{\Beta}{{\rm B}}
\newcommand{\sn}{{\rm sn}}
\newcommand{\cn}{{\rm cn}}
\newcommand{\dn}{{\rm dn}}
\newcommand{\Ec}{{\rm Ec}}
\newcommand{\Es}{{\rm Es}}
%%%%%%%%%%%%%%%%%%%%%%%%%%%%%%%%%%%%%%%%%%%%%%%%%%%%%%%%%%%%%%%%%%%%%%%%%%%%%%


\makeatletter
\newcommand\Dotfill{\leavevmode \cleaders \hb@xt@ 1.1em{\hss .\hss }\hfill \kern \z@}
\makeatother


\suppressfloats %\nofiles

\newcommand{\supp}{\mathop{{\rm supp}}}
\newcommand{\esssup}{\mathop{{\rm ess\,sup}}}


%%%%%%%%%%%%%%%%%%%%%%%%%%%%%%%%%%% NOTICE %%%%%%%%%%%%%%%%%%%%%%%%%%%%%%%%%%%
% Fill in your title and other information here.

\title{Title here}
\formattedtitle{
		DEEP LEARNING USING ATTENTION AND GRAPH MODELS \\
		FOR HIGH ENERGY PHYSICS WITH APPLICATIONS IN \\
		HIGGS ANALYSIS, TOP QUARK ENTANGLEMENT, \\
		AND PILEUP MITIGATION \\
               }

\author{LUKE MARTIN VAUGHAN}

\degreeone{\ssp Bachelor of Science in Applied Physics\\
        Oklahoma State University\\
        Stillwater, Oklahoma\\
        2021}

\degreetwo{\ssp Bachelor of Science in Aerospace and Mechanical Engineering\\
        Oklahoma State University\\
        Stillwater, Oklahoma\\
        2021}

\degreethree{\ssp Master of Science in Physics\\
        Oklahoma State University\\
        Stillwater, Oklahoma\\
        2024}

\degreesought{DOCTOR OF PHILOSOPHY}
 \degreedate{May, 2026}
\majorfield{Physics}

%%%%%%%%%%%%%%%%%%%%%%%%%%%%%%%%%%%%%%%%%%%%%%%%%%%%%%%%%%%%%%%%%%%%%%%%%%%%%%


\begin{document}

\maketitle
%\makecopyright

%%%%%%%%%%%%%%%%%%%%%%%%%%%%%%%%%%% NOTICE %%%%%%%%%%%%%%%%%%%%%%%%%%%%%%%%%%%
% Put your advisor's and the committee members' names here:
\makeapproval{Dr. Flera Rizatdinova}{Dr. Alexander Khanov}{Dr. Dorival Gonçalves}{Dr. Atriya Sen}

\begin{acknowledge}
 I would like to thank my advisor Dr.~Advisor. I would also like to thank my co-author Dr.~Coauthor. 

I would like to thank my family for their patience.

%%%%%%%%%%%%%%%%%%%%%%%%%%%%%%%%%%% NOTICE %%%%%%%%%%%%%%%%%%%%%%%%%%%%%%%%%%%
% If you Acknowledgments span MORE than one page, the graduate college
% requires that you have this footer present on every page. In order to put it
% there, you will put this footer command in some of the text on page 2 (or 3,
% etc.) in order to ensure a copy of the footer appears on each page:

%\blfootnote{Acknowledgments reflect the views of the author and are not endorsed
% by committee members or Oklahoma State University.}

%%%%%%%%%%%%%%%%%%%%%%%%%%%%%%%%%%%%%%%%%%%%%%%%%%%%%%%%%%%%%%%%%%%%%%%%%%%%%%

\end{acknowledge}

%%%%%%%%%%%%%%%%%%%%%%%%%%%%%%%%%%% NOTICE %%%%%%%%%%%%%%%%%%%%%%%%%%%%%%%%%%%
% Fields passed to abstract are:
% Name, Month and Year of Graduation, Title, Field, and spacing: Use 0in or 1in per GC policy,
% depending on length of the abstract. A longer abstract will 0in, a shorter abstract 1in.

\begin{abstract}{LUKE MARTIN VAUGHAN}
{MAY 2026}
{TITLE HERE IN ALL CAPS}
{PHYSICS}
{60pt}
   %Abstract will go here.  Make sure it remains within the 350 word limit.
%\par To get a new paragraph
Here is where your abstract should go. 
\par 
Lorem ipsum dolor sit amet, consectetur adipiscing elit. Etiam finibus venenatis dui, a accumsan dui elementum non. Suspendisse suscipit diam sed dapibus mollis. Quisque id congue nisl, auctor elementum turpis. Sed mattis at leo non rhoncus. Donec at rhoncus velit, at dignissim risus. Sed in quam a felis pulvinar bibendum a eget mi. Aliquam ac ligula nec urna pharetra interdum. Nam varius quis dui non finibus. Proin ullamcorper blandit ipsum nec feugiat.
\par 
Etiam et massa ut augue venenatis lobortis. Integer placerat libero eros, nec mattis nisl tempus nec. Cras in nisl enim. Etiam fermentum commodo ornare. Praesent scelerisque viverra rutrum. Etiam ultricies velit lacus, vel elementum ante ullamcorper viverra. Ut pretium eget ligula a facilisis. Pellentesque interdum tellus metus, vitae porttitor libero venenatis in. Nunc id velit eros. Quisque sapien neque, volutpat ut cursus nec, vehicula id turpis. Interdum et malesuada fames ac ante ipsum primis in faucibus. 

\end{abstract}
%%%%%%%%%%%%%%%%%%%%%%%%%%%%%%%%%%%%%%%%%%%%%%%%%%%%%%%%%%%%%%%%%%%%%%%%%%%%%%


\renewcommand{\listfigurename}{LIST OF FIGURES}
\renewcommand{\listtablename}{LIST OF TABLES}

{\addtocontents{toc}{\protect\renewcommand{\protect\cftchapleader}
 {\protect\cftdotfill{\cftsecdotsep}}} %<-adds leaders at the chapter level
\addtocontents{toc}{~{\kern-0.3in Chapter}\hfill{Page}\par} %<-this line deals with the first toc page

\fancypagestyle{fancyplain}{ %
 \lhead{ \fancyplain{}{Chapter} }
 \rhead{ \fancyplain{}{Page} }
 \renewcommand{\headrulewidth}{0pt} % remove lines which fancyhdr usually uses
 \renewcommand{\footrulewidth}{0pt}
}

\changepage{-29.43001pt}{}{}{}{}{}{15pt}{12pt}{}
\pagestyle{fancyplain}
\thispagestyle{plain}
\tableofcontents
}

\renewcommand{\cftfigaftersnum}{.}
\renewcommand{\cfttabaftersnum}{.}

%%% Generate the list of tables
\fancypagestyle{fancyplain}{ %
 \lhead{ \fancyplain{}{Table} }
 \rhead{ \fancyplain{}{Page} }
 \renewcommand{\headrulewidth}{0pt} % remove lines as well
 \renewcommand{\footrulewidth}{0pt}
}
\changepage{-29.43001pt}{}{}{}{}{}{15pt}{12pt}{}
\pagestyle{fancyplain}\renewcommand{\thepage}{\roman{page}}
\thispagestyle{plain}\renewcommand{\thepage}{\roman{page}}
\ssp
%%%%%%%%%%%%%%%%%%%%%%%%%%%%%%%%%%% NOTICE %%%%%%%%%%%%%%%%%%%%%%%%%%%%%%%%%%%
% Comment this line out if you do not want a List of Tables:
\listoftables
%%%%%%%%%%%%%%%%%%%%%%%%%%%%%%%%%%%%%%%%%%%%%%%%%%%%%%%%%%%%%%%%%%%%%%%%%%%%%%


%%% Generate the list of figures
\fancypagestyle{fancyplain}{ %
 \lhead{ \fancyplain{}{Figure} }
 \rhead{ \fancyplain{}{Page} }
 \renewcommand{\headrulewidth}{0pt} % remove lines as well
 \renewcommand{\footrulewidth}{0pt}
}
\pagestyle{fancyplain}\renewcommand{\thepage}{\roman{page}}
\thispagestyle{plain}\renewcommand{\thepage}{\roman{page}}
\ssp

%%%%%%%%%%%%%%%%%%%%%%%%%%%%%%%%%%% NOTICE %%%%%%%%%%%%%%%%%%%%%%%%%%%%%%%%%%%
% Comment this line out if you do not want a List of Figures:
\listoffigures
%%%%%%%%%%%%%%%%%%%%%%%%%%%%%%%%%%%%%%%%%%%%%%%%%%%%%%%%%%%%%%%%%%%%%%%%%%%%%%

\pagestyle{plain}

\changepage{27pt}{}{}{}{}{}{-15pt}{-12pt}{} % Reset headers height to zero and restore textheight
\dsp

%%%%%%%%%%%%%%%%%%%%%%%%%%%%%%%%%%% NOTICE %%%%%%%%%%%%%%%%%%%%%%%%%%%%%%%%%%%
% Optional: uncomment the following lines if you want to include a Nomenclature page,
% and create TeX file notation.tex to use.
% \begin{nomenclature}
%  \begin{tabular}{cl}
\emph{Hello} & This is a thing\\
\emph{This} & Means that\\
\end{tabular}
%\newpage

% \end{nomenclature}

%%%%%%%%%%%%%%%%%%%%%%%%%%%%%%%%%%% NOTICE %%%%%%%%%%%%%%%%%%%%%%%%%%%%%%%%%%%
% Optional: uncomment the following 3 lines if you want to include a list of symbols page
% \begin{listofsymbols}
%  \input{misc/listofsymbols}
% \end{listofsymbols}

\newpage
\pagenumbering{arabic}\setcounter{page}{1}

%%%%%%%%%%%%%%%%%%%%%%%%%%%%%%%%%%% NOTICE %%%%%%%%%%%%%%%%%%%%%%%%%%%%%%%%%%%
% This is where you will put your thesis chapters files. You do not need to
% include the .tex extension, but this input command is inputing chapter1.tex,
% etc.

\begingroup
\clearpage% Manually insert \clearpage
\let\clearpage\relax% Remove \clearpage functionality
\vspace*{-16pt}% Insert needed vertical retraction
\chapter[INTRODUCTION]{INTRODUCTION}\label{chap:intro}
\endgroup

\section{Outline}

\begin{itemize}
  \item general introduction, breif history of QM and particle physics
  \item Explain HEP research focus during my time as PhD such higgs theory and analysis.
  \item How does this research fit into global picture.
  \item Modern statistical analysis methods in HEP (ML: Adversarial, Graph, Attention).
  \item HL-LHC mention/motivation
\end{itemize}

\begin{figure}
  \includegraphics[width=\linewidth]{figures/chapter1/LHC-and-HL-LHC-Schedule.png}
  \caption{The upgrade shcedule for HL-LHC operations. \cite{Zerlauth:2024PD}}
\end{figure}

\begin{figure}[t]
    \centering
    \begin{subfigure}{.49\textwidth}
      \centering
      \includegraphics[width=\linewidth]{figures/chapter1/Event_30_mu60.png}
      \caption{}
      \label{fig:Efrac1d_mu60}
    \end{subfigure}\hfill
    \begin{subfigure}{.49\textwidth}
      \centering
      \includegraphics[width=\linewidth]{figures/chapter1/Event_30_mu200.png}
      \caption{}
      \label{fig:Efrac2d_mu60}
    \end{subfigure}\hfill
    \caption{A Phoenix Event Display\cite{phoenix} depicting an event in the ATLAS inner tracking system in the HL-LHC conditions at $\langle\mu\rangle=60$ (left) and $\langle\mu\rangle=200$ (right) for signal (blue) and background (red) particles with $p_T>1.0$ GeV.}
    \label{fig:PhoenixDisplay}
\end{figure}


\begingroup
\clearpage% Manually insert \clearpage
\let\clearpage\relax% Remove \clearpage functionality
\vspace*{-16pt}% Insert needed vertical retraction
\chapter[THE STANDARD MODEL OF PARTICLE PHYSICS]{THE STANDARD MODEL OF PARTICLE PHYSICS}
\endgroup

\section{Outline}

\begin{itemize}
  \item Standard model is huge success but there are open problems
  \item QED and QCD
  \item Feynman diagrams that describe VH and VBF bb/cc verticies with lagrangian
  \item we prove these couplings, and deviations would demonstrate new physics
  \item Higgs yukawa couplings to third and second generations. tH and signs of kappa wrt VH
  \item Why is this physics important? Why are these channels important?
  \item Why is this important to stability of the universe? Higgs potential.
  \item A few things beyond VBF Higgs
\end{itemize}


\begingroup
\clearpage% Manually insert \clearpage
\let\clearpage\relax% Remove \clearpage functionality
\vspace*{-16pt}% Insert needed vertical retraction
\chapter[THE ATLAS DETECTOR]{THE ATLAS DETECTOR}
\endgroup

\section{Outline}

\begin{itemize}
  \item Standard description of detector
  \item Special attenion to tracker and calorimeter
  \item btagging - and why its important to tag b and c jets
  \item pflow jets and reconstruction algorithms
\end{itemize}



\begingroup
\clearpage% Manually insert \clearpage
\let\clearpage\relax% Remove \clearpage functionality
\vspace*{-16pt}% Insert needed vertical retraction
\chapter[FOURTH CHAPTER TITLE]{FOURTH CHAPTER TITLE}
\endgroup

Here we start Chapter IV.
\section{More sections}
Here is some text.

\section{More sections}
Here is some text.

\subsection{Another subsection}
Here is some text.

\subsection{Another subsection}
Here is some text.
\subsubsection{Another subsection}
Here is some text.

\subsubsection{Another}
Again!

\subsection{Another subsection}
Here is some text.



\begingroup
\clearpage% Manually insert \clearpage
\let\clearpage\relax% Remove \clearpage functionality
\vspace*{-16pt}% Insert needed vertical retraction
\chapter[CALIBRATION OF GN2Xv01 BOOSTED JET TAGGER]{CALIBRATION OF GN2Xv01 BOOSTED JET TAGGER}
\endgroup

\section{Key Contributions}

\begin{itemize}
  \item Derived ttbar enriched region using ATLAS AnalysisTop framework in ATHENA r24.
  \item Validated GN2X discriminant in ttbar enriched region and quantified ttbar mistag rate.
  \item Derived scale factors to account for the differences between MC and data.

\end{itemize}

\section{Boosted Jet Tagging with Graph and Attention Neural Networks}

\lipsum[1]

\begin{align}
	D^{GN2X}_{Hbb} &= ln \left( \frac{p_{Hbb}}{f_{Hcc} \cdot p_{Hcc} + f_{top} \cdot p_{top} + (1-f_{Hcc}-f_{top})\cdot p_{QCD}} \right) \\
	D^{GN2X}_{Hcc} &= ln \left( \frac{p_{Hcc}}{f_{Hbb} \cdot p_{Hbb} + f_{top} \cdot p_{top} + (1-f_{Hbb}-f_{top})\cdot p_{QCD}} \right)
\end{align}

\lipsum[1]

\begin{figure}
	\centering
	\includegraphics[width=0.5\linewidth]{figures/chapter5/DXbb.png}
	\caption{}
	\label{fig:Yields}
\end{figure}

\section{ttbar Mistag Calibration}

\lipsum[1]

\begin{equation}
	SF=\frac{\epsilon^{data}}{\epsilon^{MC}}
\end{equation}

\lipsum[1]

\begin{align}
	N^{data}_{tag} &= \mu \cdot \frac{\epsilon^{data}}{\epsilon^{MC}} \cdot N^{ttbar}_{tag} + N^{other}_{tag} \\
	N^{data}_{untag} &= \mu \cdot \frac{1-\epsilon^{data}}{1-\epsilon^{MC}} \cdot N^{ttbar}_{untag} + N^{other}_{untag}
\end{align}

Or equivalently the formula can be written in terms of scale factors:

\begin{align}
	N^{data}_{tag} &= \mu \cdot SF \cdot N^{ttbar}_{tag} + N^{other}_{tag} \\
	N^{data}_{untag} &= \mu \cdot \frac{1-\epsilon^{MC} \cdot SF}{1-\epsilon^{MC}} \cdot N^{ttbar}_{untag} + N^{other}_{untag}
\end{align}

\lipsum[1]

\subsection{Event Selection}

\lipsum[1]

\begin{figure}
	\centering
	\includegraphics[width=0.7\linewidth]{figures/chapter5/Reco_Event.png}
	\caption{}
	\label{fig:Yields}
\end{figure}

\lipsum[1]

\begin{table}[h]
	\centering
	\resizebox{0.7\textwidth}{!}{%
		\begin{tabular}{ccc}
			\hline
			\textbf{Cut Applied} & \textbf{Electron Channel} & \textbf{Muon Channel} \\
			\hline
			Initial Events                   & 8,911,000 & 8,911,000 \\
			Primary Vertex                   & 8,910,952 & 8,910,952 \\
			Reconstruction                   & 1,220,162 & 1,220,162 \\
			Lepton $p_T > 25$ GeV            & 374,572   & 400,935   \\
			Tight lepton $> 70$ GeV          & 96,701    & 79,744    \\
			Pflow jet $p_T > 25$ GeV         & 94,826    & 77,999    \\
			UFO large-R jet $p_T > 200$ GeV  & 53,462    & 41,511    \\
			MET                              & 44,089    & 33,959    \\
			\hline
		\end{tabular}%
	}
	\caption{Cutflow table for electron and muon selections.}
	\label{tab:cutflow}
\end{table}

%\begin{figure}
%	\centering
%	\includegraphics[width=0.7\linewidth]{figures/chapter5/Cutflow.png}
%	\caption{}
%	\label{fig:Yields}
%\end{figure}

\lipsum[1-3]

\begin{figure}
	\centering
	\begin{subfigure}{.24\textwidth}
		\centering
		\includegraphics[width=\linewidth]{figures/chapter5/probe_jet_pt.png}
		\caption{}
		\label{fig:sub1}
	\end{subfigure}%
	\begin{subfigure}{.24\textwidth}
		\centering
		\includegraphics[width=\linewidth]{figures/chapter5/probe_jet_eta.png}
		\caption{}
		\label{fig:sub2}
	\end{subfigure}
	\begin{subfigure}{.24\textwidth}
		\centering
		\includegraphics[width=\linewidth]{figures/chapter5/probe_jet_phi.png}
		\caption{}
		\label{fig:sub2}
	\end{subfigure}
	\begin{subfigure}{.24\textwidth}
	\centering
	\includegraphics[width=\linewidth]{figures/chapter5/probe_jet_mass.png}
	\caption{}
	\label{fig:sub2}
	\end{subfigure}
	\caption{}
	\label{fig:ANN_Scores}
	\begin{subfigure}{.24\textwidth}
		\centering
		\includegraphics[width=\linewidth]{figures/chapter5/probe_jet_mass_300_400.png}
		\caption{}
		\label{fig:sub1}
	\end{subfigure}%
	\begin{subfigure}{.24\textwidth}
		\centering
		\includegraphics[width=\linewidth]{figures/chapter5/probe_jet_mass_400_500.png}
		\caption{}
		\label{fig:sub2}
	\end{subfigure}
	\begin{subfigure}{.24\textwidth}
		\centering
		\includegraphics[width=\linewidth]{figures/chapter5/probe_jet_mass_500_600.png}
		\caption{}
		\label{fig:sub2}
	\end{subfigure}
	\begin{subfigure}{.24\textwidth}
		\centering
		\includegraphics[width=\linewidth]{figures/chapter5/probe_jet_mass_600_1000.png}
		\caption{}
		\label{fig:sub2}
	\end{subfigure}
	\caption{}
	\label{fig:ANN_Scores}
\end{figure}

\subsection{Constructing Fit Regions}

\lipsum[1]

\begin{figure}
	\centering
	\begin{subfigure}{.32\textwidth}
		\centering
		\includegraphics[width=\linewidth]{figures/chapter5/CR_pt_200_postfit.png}
		\caption{}
		\label{fig:sub1}
	\end{subfigure}%
	\begin{subfigure}{.32\textwidth}
		\centering
		\includegraphics[width=\linewidth]{figures/chapter5/CR_pt_250_postfit.png}
		\caption{}
		\label{fig:sub2}
	\end{subfigure}
	\begin{subfigure}{.32\textwidth}
		\centering
		\includegraphics[width=\linewidth]{figures/chapter5/CR_pt_300_postfit.png}
		\caption{}
		\label{fig:sub2}
	\end{subfigure}
	\begin{subfigure}{.32\textwidth}
		\centering
		\includegraphics[width=\linewidth]{figures/chapter5/CR_pt_400_postfit.png}
		\caption{}
		\label{fig:sub2}
	\end{subfigure}
	\begin{subfigure}{.32\textwidth}
	\centering
	\includegraphics[width=\linewidth]{figures/chapter5/CR_pt_500_postfit.png}
	\caption{}
	\label{fig:sub2}
	\end{subfigure}
	\begin{subfigure}{.32\textwidth}
	\centering
	\includegraphics[width=\linewidth]{figures/chapter5/CR_pt_600_postfit.png}
	\caption{}
	\label{fig:sub2}
	\end{subfigure}
	\caption{}
	\label{fig:ANN_Scores}
\end{figure}

\lipsum[1]

\begin{figure}
	\centering
	\begin{subfigure}{.32\textwidth}
		\centering
		\includegraphics[width=\linewidth]{figures/chapter5/SR_pt_200_postfit.png}
		\caption{}
		\label{fig:sub1}
	\end{subfigure}%
	\begin{subfigure}{.32\textwidth}
		\centering
		\includegraphics[width=\linewidth]{figures/chapter5/SR_pt_250_postfit.png}
		\caption{}
		\label{fig:sub2}
	\end{subfigure}
	\begin{subfigure}{.32\textwidth}
		\centering
		\includegraphics[width=\linewidth]{figures/chapter5/SR_pt_300_postfit.png}
		\caption{}
		\label{fig:sub2}
	\end{subfigure}
	\begin{subfigure}{.32\textwidth}
		\centering
		\includegraphics[width=\linewidth]{figures/chapter5/SR_pt_400_postfit.png}
		\caption{}
		\label{fig:sub2}
	\end{subfigure}
	\begin{subfigure}{.32\textwidth}
		\centering
		\includegraphics[width=\linewidth]{figures/chapter5/SR_pt_500_postfit.png}
		\caption{}
		\label{fig:sub2}
	\end{subfigure}
	\begin{subfigure}{.32\textwidth}
		\centering
		\includegraphics[width=\linewidth]{figures/chapter5/SR_pt_600_postfit.png}
		\caption{}
		\label{fig:sub2}
	\end{subfigure}
	\caption{}
	\label{fig:ANN_Scores}
\end{figure}

\lipsum[1]

\subsection{Results}

\lipsum[1]

\begin{figure}
	\centering
	\begin{subfigure}{.49\textwidth}
		\centering
		\includegraphics[width=\linewidth]{figures/chapter5/scale_factors_mc16.png}
		\caption{}
		\label{fig:sub1}
	\end{subfigure}%
	\begin{subfigure}{.49\textwidth}
		\centering
		\includegraphics[width=\linewidth]{figures/chapter5/scale_factors_mc20.png}
		\caption{}
		\label{fig:sub2}
	\end{subfigure}
	\caption{}
	\label{fig:ANN_Scores}
\end{figure}

\lipsum[1]

\section{conclusion}

\lipsum[1]



\begingroup
\clearpage% Manually insert \clearpage
\let\clearpage\relax% Remove \clearpage functionality
\vspace*{-16pt}% Insert needed vertical retraction
\chapter[PILEUP MITIGATION AT THE HL-LHC]{PILEUP MITIGATION AT THE HL-LHC}
\endgroup

\section{Outline}

\begin{itemize}
  \item HL-LHC shutdown
  \item Next 3 years detector will undergo huge transformation
  \item Huge amount of pileup
  \item This work benefits entire experiment. Not for a specific analysis.
  \item Important to address pileup to maximize the discovery potential of the ATLAS detector at HL-LHC conditions.
\end{itemize}



\begingroup
\clearpage% Manually insert \clearpage
\let\clearpage\relax% Remove \clearpage functionality
\vspace*{-16pt}% Insert needed vertical retraction
\chapter[TOP POLARIMETRY]{TOP POLARIMETRY}
\endgroup

\section{Outline}

\begin{itemize}
  \item Top polarimetry motivation and results
\end{itemize}



\input{chapters/chapter8}
\begingroup
\clearpage% Manually insert \clearpage
\let\clearpage\relax% Remove \clearpage functionality
\vspace*{-16pt}% Insert needed vertical retraction
\chapter[CONCLUSION]{CONCLUSION}
\endgroup

\section{Outline}

\begin{itemize}
  \item One day there will be conclusions...
\end{itemize}



% Include as many chapters as you would like! Add other lines like the following:
%\input{anotherchapter}
%\input{conclusions}
% But don't forget to make a TeX file for each in the same folder!

%%%%%%%%%%%%%%%%%%%%%%%%%%%%%%%%%%% NOTICE %%%%%%%%%%%%%%%%%%%%%%%%%%%%%%%%%%%
% This template is set up to use BiBTeX data and automatically generate the
% bibliography based ONLY on what references you cite, so you can have a large
% bib file and only pull what you need.
%
% If, however, you want all references in your BiBTeX file to be listed in the
% references section, EVEN IF THEY WERE NOT SPECIFICALLY CITED SOMEWHERE IN
% THE DISSERTATION, then you must uncomment the following line:

%\nocite{*}

%%%%%%%%%%%%%%%%%%%%%%%%%%%%%%%%%%%%%%%%%%%%%%%%%%%%%%%%%%%%%%%%%%%%%%%%%%%%%%

%%% You can use BibTeX generated references, or type references out yourself.
%%% You can download BibTeX information from MathSciNet and 
%%% insert it into the references.bib file. 

\bibliographystyle{osustyle.bst} % Other common styles: abbrv, amsalpha, ieeetr, alpha
\renewcommand\bibname{REFERENCES}

% Comment out the next line if you wish to do a manual bibliography:
\begingroup
\clearpage% Manually insert \clearpage
\let\clearpage\relax% Remove \clearpage functionality
\vspace*{-16pt}% Insert needed vertical retraction
\bibliography{references.bib}
\endgroup

% Uncomment the rest of this for a manual bibliography:
% \begin{thebibliography}{bg}
% \interlinepenalty=10000
% \bibitem{K1}
% R.  Adler, The Geometry of Random Fields, Wiley, Chichester, 1981.
% 
% \bibitem{ApostolAnalyticBook}
% V. Andrievskii, Weighted polynomial inequalities in the complex plane, J. Approx. Theory, 164 (2012), 1165--1183.
% 
% \bibitem{LiCai}
% A. Granville and I. Wigman, The zeros of random trignometric polynomials, Amer. J. Math. 133 (2011), 295--357.
% 
% \bibitem{HM}
% J. Hammersley, The zeros of a random polynomial, Proc. of the Third Berk. Sym. on Math. Stat. and Prob. 1954-1955 vol. II, University of Cal. Press, Berkeley and Los Angeles (1956) 89--111.
% 
% \end{thebibliography}

%%%%%%%%%%%%%%%%%%%%%%%%%%%%%%%%%%% NOTICE %%%%%%%%%%%%%%%%%%%%%%%%%%%%%%%%%%%
% You may comment out the following line if you do not have any appendices:
\appendix
%\include{chapter-appendix}

\begingroup
\clearpage% Manually insert \clearpage
\let\clearpage\relax% Remove \clearpage functionality
\vspace*{-16pt}% Insert needed vertical retraction
\chapter*{\centerline{APPENDICES}}
\endgroup

% \begin{center}
% {\bfseries Title of Appendix (Not Numbered)}
% \end{center}

\section*{Title of Appendix (Not Numbered)}
\addcontentsline{toc}{chapter}{APPENDICES}
\renewcommand{\thechapter}{A}
\numberwithin{equation}{chapter}
\setcounter{equation}{0}
%\setcounter{theorem}{0}


Lorem ipsum dolor sit amet, consectetur adipiscing elit. Etiam finibus venenatis dui, a accumsan dui elementum non. Suspendisse suscipit diam sed dapibus mollis. Quisque id congue nisl, auctor elementum turpis. Sed mattis at leo non rhoncus. Donec at rhoncus velit, at dignissim risus. Sed in quam a felis pulvinar bibendum a eget mi. Aliquam ac ligula nec urna pharetra interdum. Nam varius quis dui non finibus. Proin ullamcorper blandit ipsum nec feugiat.

%%%%%%%%%%%%%%%%%%%%%%%%%%%%%%%%%%%%%%%%%%%%%%%%%%%%%%%%%%%%%%%%%%%%%%%%%%%%%%

\newpage
%%% THE VITA CAN BE ONLY ONE PAGE IN LENGTH
 \begin{vita}{Your Name Here}{Doctor of Philosophy}{Mathematics} %Creates vita

 \vitaitem{Education:} \\
 \\
Completed the requirements for the Doctor of Philosophy in Mathematics at Oklahoma State University, Stillwater,  Oklahoma in May, 2020.\\
\\
Completed the requirements for the Master of Science in Mathematics at Other University, City,  State in 2015.\\
\\
Completed the requirements for the Bachelor of Science in Mathematics at Undergraduate University, City, State in 2011.\\

%  \vitaitem{Experience:} \\
%  \\
%  Job here
% Include this if needed:
 \vitaitem{Professional Membership:}\\
 \\
 American Mathematical Society, Mathematical Association of America

 \end{vita}


\end{document}
